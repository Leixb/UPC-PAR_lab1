% vim: spell:spelllang=en:
\input{preamble}

\renewcommand\theadfont{\bfseries}

\title{
    PAR Laboratory Assignment\\
    Lab 1: Experimental setup and tools
}

\author{
    par2109:
    Aleix Boné,
    Alex Herrero
}

\date{
    Spring 2019-20
}

\begin{document}

%After the last session for this laboratory assignment, and before starting the
%next one, you will have to deliver a report in PDF format (other formats will
%not be accepted) describing the results and conclusions that you have obtained
%when doing the assignment. As part of the document, you will have to include
%any code fragment, figure or plot you need to support your explanations. Your
%professor will open the assignment at the Raco website and set the appropriate
%delivery dates for the delivery. Only one file has to be submitted per group
%through the Raco website.
%
%Important: In the front cover of the document, please clearly state the name of
%all components of the group, the identifier of the group (username parXXYY),
%title of the assignment, date, academic

\maketitle

\section{Node architecture and memory}%
\label{sec:node_architecture_and_memory}

%Describe the architecture of the boada server. To accompany your description,
%you should refer to the following table summarising the relevant architectural
%characteristics of the different node types available:

\begin{table}[htpb]%
    \label{tab:node_arch_and_mem}
    \centering
    %\caption{caption}
    \begin{tabular}{lccc}

    \toprule
        & \texttt{boada-1 to boada-4} & \texttt{boada-5} & \texttt{boada-6 to boada-8} \\
    \midrule
        Number of sockets per node          & 2        & 2        & 2        \\
        Number of cores per socket          & 6        & 6        & 8        \\
        Number of threads per core          & 2        & 2        & 1        \\
        Maximum core frequency              & 2395 MHz & 2600 MHz & 1700 MHz \\
    \addlinespace[1em]
        L1-I cache size (per-core)          & 32K      & 32K      & 32K      \\
        L1-D cache size (per-core)          & 32K      & 32K      & 32K      \\
        L2 cache size (per-core)            & 256K     & 256K     & 256K     \\
        Last-level cache size (per-socket)  & 12288K   & 15360K   & 20480K   \\
    \addlinespace[1em]
        Main memory size (per socket)       & 12 Gb    & 31 Gb    & 16 Gb    \\
        Main memory size (per node)         & 23 Gb    & 63 Gb    & 31 Gb    \\
    \bottomrule

    \end{tabular}
\end{table}

%Also include in the description the architectural diagram for one of the nodes
%boada-1 to boada-4 as obtained when using the lstopo command. Appropriately
%comment whatever you consider appropriate.

\section{Strong vs.\ weak scalability}%
\label{sec:strong_vs_weak_scalability}

%Briefly explain what strong and weak scalability refer to. Exemplify your
%explanation using the execution time and speed–up plots that you obtained for
%pi omp.c. Reason about the results obtained.

\section{Analysis of task decompositions for \emph{3DFFT}}%
\label{sec:analysis_of_task_decompositions_for_3dfft}

%In this part of the report you should summarise the main conclusions from the
%analysis of task decompositions for the 3DFFT program. Backup your conclusions
%with the following table properly filled in with the information obtained in
%the laboratory session for the initial and different versions generated

\begin{table}[htpb]%
    \label{tab:parallelism}
    \centering
    %\caption{caption}
    \begin{tabular}{cccc}
    \toprule
    \thead{Version} & $T_1$ & $T_\infty$ & \thead{Parallelism} \\
    \midrule
    seq     & a & b & c \\
    v1      & a & b & c \\
    v2      & a & b & c \\
    v3      & a & b & c \\
    v4      & a & b & c \\
    v5      & a & b & c \\
    \bottomrule
    \end{tabular}
\end{table}

%is expected. For that include a plot with the execution time and/or speedup
%when using 1, 2, 4, 8, 16 and 32 processors, as reported by the simulation
%module inside Tareador. You should also include the relevant(s) part(s) of the
%code that help the reader to understand why v5 is able to scale to a higher
%number of processors compared to v4, capturing the task dependence graphs that
%are obtained with

\section{Understanding the parallel execution of \emph{3DFFT}}%
\label{sec:understanding_the_parallel_execution_of_3dfft}

%In this final section of your report you should comment about how did you
%observed with Paraver the parallel performance evolution for the OpenMP
%parallel versions of 3DFFT. Support your explanations with the results reported
%in the following table which you obtained during the laboratory session. It is
%very important that you include the relevant Paraver captures (timelines and
%profiles of the % of time

\begin{table}[htpb]%
    \label{tab:under_parallelism}
    \centering
    %\caption{caption}
    \begin{tabular}{lcc@{\hskip 2em}ccc}
    \toprule
    \thead{Version} & $\phi$ & $S_\infty$ & $T_1$ & $T_8$ & $S_8$ \\
    \midrule
    initial version in \texttt{3dfft\_omp.c}                & phi & Sinf & t1 & t8 & S8 \\
    new version with improved $\phi$                        & phi & Sinf & t1 & t8 & S8 \\
    final version with reduced parallelization overheads    & phi & Sinf & t1 & t8 & S8 \\
    \bottomrule
    \end{tabular}
\end{table}

%Finally you should comment about the (strong) scalability plots (execution time
%and speed–up) that are obtained when varying the number of threads for the
%three parallel versions that you have analysed.

\end{document}
